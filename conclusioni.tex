\chapter*{Conclusioni}
\addcontentsline{toc}{chapter}{Conclusioni}
L'enorme quantitativo di informazioni contenute all'interno di una TC, una tipologia di esame estremamente versatile e richiesta in gran parte delle specialità mediche per i motivi più disparati, può tornare utile per la stima della composizione corporea dei singoli pazienti. In particolare, misurazioni effettuate mediante TC nella regione lombare sono ritenute altamente rappresentative della composizione corporea del singolo paziente ed è ampiamente dimostrato il loro valore prognostico, soprattutto in caso di tumori. Si potrebbe affermare, in realtà, che i metodi fondati sulla TC non sono solo utili ma quasi necessari, nell'ottica di un progressivo cambiamento degli attuali protocolli per la valutazione della densità minerale ossea, della quantità di grasso, della carenza muscolare e in generale di tutti i metodi mirati, per un motivo o per un altro, alla stima della \textit{body composition}.

È doveroso aggiungere, però, un'importante precisazione: accanto a situazioni in cui i protocolli odierni possono condurre a valutazioni errate dello stato delle cose, e in cui sarebbe invece fondamentale per la salute e la vita del paziente avere informazioni il più precise possibile (è il caso dell'indice di massa corporea usato come estimatore della \textit{body composition} in fase di pianificazione della terapia oncologica), ci sono altre situazioni in cui i metodi odierni non offrono risultati tanto peggiori rispetto alla TC: è il caso, per fare un esempio, della DXA per la valutazione della densità minerale ossea e la diagnosi dell'osteoporosi. In altri casi ancora è stata riscontrata la non adeguatezza degli standard attuali ma neanche della TC, come nel caso di alcuni pazienti colpiti da cancro al polmone, per i quali le grandezze \textit{CT-derived} non sono risultate correlate con la sopravvivenza dei pazienti stessi. In sintesi, la TC è un esame che va valutato con attenzione ma che acquista un enorme valore all'interno dell'approccio di \textit{screening} opportunistico, che prevede l'estrazione di tutti i dati che possono tornare utili da TC che vengono acquisite per altre motivazioni cliniche.

Tuttavia, questi dati possono essere sfruttati a pieno solo mediante metodi di segmentazione automatici, i quali permettono di calcolare quantità come densità, superfici e indici dei diversi tessuti corporei che altrimenti sarebbero impossibili da stimare. I software attualmente disponibili, in particolare le reti neurali a convoluzione, riescono a produrre segmentazioni di immagini TC che dimostrano un ottimo accordo con le segmentazioni effettuate manualmente da radiologi e anatomisti. Un problema da tenere in considerazione durante la progettazione e l'allenamento di un software di questo tipo è l'effetto del mezzo di contrasto sulle grandezze \textit{CT-derived}, che pare essere importante, sebbene al momento non siano presenti studi a sufficienza per avere un'idea consolidata dell'entità del problema.

Infine, un altro importante aspetto della segmentazione dei tessuti in immagini TC per la stima della composizione corporea è l'approccio 3D in contrapposizione al metodo \textit{single slice}: una valutazione della \textit{body composition} sui volumi piuttosto che sulle superfici, infatti, riesce a ottenere risultati molto più accurati, perché permette di superare il problema della rappresentatività delle singole \textit{CT slice}, che per quanto alta (in base al tessuto che consideriamo) non potrà mai essere rappresentativa al 100\% della composizione tissutale di tutto il corpo. L'approccio 3D non potrebbe ovviamente essere praticabile senza la segmentazione automatica, rappresentando un motivo in più per lo sviluppo di software che implementino questa modalità di segmentazione.