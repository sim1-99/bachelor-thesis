\begin{abstract}
Segmentare un’immagine significa riconoscere al suo interno elementi con caratteristiche comuni e raggrupparli, distinguendoli dagli elementi che posseggono caratteristiche diverse; si parla di segmentazione automatica quando questo processo è eseguito completamente da un software senza l’intervento umano. Segmentare le immagini TC, molto diffuse in ambito diagnostico, permette di estrarre una grande quantità di dati dall’alto valore prognostico e predittivo della composizione corporea del paziente. Tuttavia, a causa della scarsa diffusione di software per la segmentazione automatica, tutti questi dati non vengono utilizzati.

Il presente lavoro di tesi si propone di riportare lo stato dell’arte sulla segmentazione, sia manuale sia automatica, dei tessuti corporei in immagini TC. Vengono spiegati i vantaggi dell’utilizzo di grandezze \textit{CT-derived} rispetto a molti dei protocolli odierni e vengono esposti gli attuali livelli di accuratezza delle segmentazioni effettuate con metodi automatici. Inoltre, ci si sofferma, cercando di quantificarli, sugli effetti del mezzo di contrasto sulle grandezze \textit{CT-derived}, poiché questi possono generare errori nella segmentazione automatica dei tessuti. Infine, viene esposto l’approccio 3D alla segmentazione in contrapposizione al metodo \textit{single slice}, con il primo caratterizzato da un’accuratezza maggiore del secondo.
\end{abstract}