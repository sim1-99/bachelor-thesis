\chapter*{Introduzione}
\addcontentsline{toc}{chapter}{Introduzione}
La tomografia computerizzata (TC) è una tecnica diagnostica fondata, come la radiografia, sull'assorbimento di raggi X, ma rispetto alla tradizionale radiografia proiettiva, la TC ha il vantaggio di generare immagini volumetriche. La TC riesce a mettere in evidenza dettagli che altrimenti non sarebbe possibile notare da una semplice serie di immagini radiografiche proiettive. Un'altra differenza fondamentale è il passaggio dal pixel al voxel, il che risolve le ambiguità dovute alla sovrapposizione di diversi oggetti lungo una direzione, non distinguibili in radiografia. Il principio di ricostruzione dell'immagine nella TC è piuttosto semplice: si tratta di acquisire immagini radiografiche da direzioni diverse e metterle insieme per individuare il punto nello spazio in cui è collocato l'oggetto che ha generato l'attenuazione. Ciò è possibile grazie alla rotazione completa del \textit{gantry}, cioè il dispositivo che contiene il tubo a raggi X, in una traiettoria che risulta essere ad elica attorno al paziente, posto su un lettino motorizzato.

La TC è una tecnica diagnostica incredibilmente versatile, infatti può essere utilizzata per diagnosticare diverse patologie o effettuare valutazioni cliniche, come lo studio della densità ossea, correlata all'osteoporosi, la stima del grado di obesità e di sarcopenia del paziente. Come altre tecniche radiografiche anche in TC è possibile utilizzare mezzi di contrasto per poter evidenziare \textit{uptake} di strutture di interesse. Possedere informazioni su queste condizioni cliniche è importante per la valutazione dei fattori di rischio, per la formulazione di strategie per ridurlo e per il loro valore prognostico in persone già affette da altre malattie.

Spesso le informazioni delle indagini TC sono acquisite per scopi non diagnostici, come nel caso della PET/TC o della TC di centratura radioterapica. Queste informazioni sono preziose per studiare i volumi dei tessuti corporei evitando di esporre il paziente a esami specifici, senza costi aggiuntivi per la struttura ospedaliera né ulteriore assorbimento di radiazioni al paziente stesso. Quasi mai queste informazioni vengono utilizzate, sebbene siano preziose perché possono essere usate per classificare i tessuti del paziente estratti dalle immagini mediante un procedimento noto come segmentazione.

Segmentare un’immagine significa riconoscere al suo interno elementi con caratteristiche in comune, distinguere questi elementi da altri con caratteristiche diverse e raggruppare tutti gli elementi simili in regioni, delineando dei bordi tra di esse. In un’immagine digitale, la segmentazione consiste nel classificare e quantificare in qualche modo le proprietà di ciascun pixel, come l’intensità, il colore e la \textit{texture}. Sebbene possa sembrare un’operazione semplice, effettuare manualmente un lavoro di segmentazione su una serie di immagini TC è un processo ripetitivo, operatore-dipendente e che richiede parecchio tempo. Da qui nasce l’esigenza di un metodo automatizzato, fondato sull'utilizzo di reti neurali, per segmentare i diversi tessuti del corpo umano.

Lo scopo di questa tesi è quello di riportare lo stato dell'arte sulla segmentazione, sia manuale sia automatica, dei tessuti corporei in immagini TC, spiegando tutti i vantaggi dell'utilizzo di grandezze \textit{CT-derived} rispetto a molti dei protocolli odierni ed esponendo gli odierni livelli di accuratezza delle segmentazioni effettuate con metodi automatici. Inoltre, ci si sofferma, cercando di quantificarli, sugli effetti del mezzo di contrasto sulle grandezze \textit{CT-derived}, poiché questi possono generare errori nella segmentazione automatica dei tessuti. Infine, viene esposto l'approccio 3D alla segmentazione in contrapposizione al metodo \textit{single slice}, dove il primo è caratterizzato da un'accuratezza maggiore del secondo.

L'esposizione degli argomenti è articolata come segue.
\begin{description}
\item[Il primo capitolo] contiene le informazioni preliminari necessarie per comprendere a pieno il contenuto dei capitoli successivi, quali la tecnologia della tomografia computerizzata, le basi teoriche della segmentazione, un'introduzione alle reti neurali, in particolare quelle a convoluzione, e alcuni metodi statistici.
\item[Il secondo capitolo] espone gli obiettivi e l'utilità dello studio della composizione corporea, illustrando anche i vantaggi derivanti dall'utilizzo di immagini TC e di grandezze \textit{CT-derived} rispetto ai protocolli odierni di stima della \textit{body composition}; contemporaneamente viene sottolineata la necessità di metodi di segmentazione automatica per permettere il pieno utilizzo delle informazioni contenute all'interno delle TC.
\item[Il terzo capitolo,] infine, contiene lo stato dell'arte sulla segmentazione automatica dei tessuti corporei, una panoramica sugli effetti del mezzo di contrasto sulle grandezze \textit{CT-derived} e un accenno all'approccio 3D per la stima della \textit{body composition}.
\end{description}


